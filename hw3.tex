\documentclass[letterpaper]{article}

%\usepackage{amsfonts}
\usepackage{amsthm,amsmath,amssymb}
\usepackage[utf8]{inputenc}
\usepackage{polynom}
\usepackage{hhline}

\title{CS 6001 Homework 3}
\author{Michael Catanzaro, Jacob Fischer, Christian Storer}
\date{October 21, 2016}

\begin{document}
\maketitle

\section{Problem 1}

% https://engineering.purdue.edu/kak/compsec/NewLectures/Lecture6.pdf

\((9x^2+3x+5) / (7x+3)\)

\bigskip

\begin{tabular}{rrrrr}
     &   & 6x       & + 1  & R 2 \\ \hhline{~----}
7x+3 & ) & \(9x^2\) & + 3x & + 5 \\
     & - & \(9x^2\) & + 7x &     \\ \hhline{~~---}
     &   &          & 7x   & + 5 \\
     &   & -        & 7x   & + 3 \\ \hhline{~~~--}
     &   &          &      & 2 
\end{tabular}

\((9x^2+3x+5) / (7x+3) = 6x + 1,\ R\ 2\)

% Note: This answer seems wrong to me

\section{Problem 2}

\subsection{Addition}

\((x^5 + x^3 + x^2 + x + 1)\ +\ (x^2 + x + 1)\)

= \(x^5 + x^3\)

\subsection{Subtraction}

\((x^5 + x^3 + x^2 + x + 1)\ -\ (x^2 + x + 1)\)

= \(x^5 + x^3\)

\subsection{Multiplication}

\((x^5 + x^3 + x^2 + x + 1)\ *\ (x^2 + x + 1)\)

\begin{equation*}
  \begin{split}
    x^5 + x^3 + x^2 + x + 1 * x^2 &= x^7 + x^5 + x^4 + x^3 + x^2 \\
    x^5 + x^3 + x^2 + x + 1 * x^2 &= x^6 + x^4 + x^3 + x^2 + x \\
    x^5 + x^3 + x^2 + x + 1 * 1 &= x^5 + x^3 + x^2 + x + 1
  \end{split}
\end{equation*}

\begin{tabular}{ c c c c c c c c }
\(x^7\) &   & \(+\ x^5\) & \(+\ x^4\) & \(+\ x^3\) &\(+\ x^2\)&   &   \\
  & \(+\ x^6\) &   & \(+\ x^4\) & \(+\ x^3\) & \(+\ x^2\) &\(+\ x\) &   \\
  &   & \(+\ x^5\) &   & \(+\ x^3\) & \(+\ x^2\) & \(+\ x\) & \(+\ 1\) \\
\end{tabular}

= \(x^7 + x^6 + x^3 + x^2 + 1\)

\subsection{Division}

\((x^5 + x^3 + x^2 + x + 1)\ /\ (x^2 + x + 1)\)

\bigskip

\begin{tabular}{rrrrrrrr}
                &   & \(x^3\) & + \(x^2\) & + x       & + 1       &     & R x \\ \hhline{~-------}
\(x^2\) + x + 1 & ) & \(x^5\) &           & + \(x^3\) & + \(x^2\) & + x & + 1 \\
                & - & \(x^5\) & + \(x^4\) & + \(x^3\) &           &     &     \\ \hhline{~~------}
                &   &         &   \(x^4\) &           & + \(x^2\) & + x & + 1 \\
                &   &       - &   \(x^4\) & + \(x^3\) & + \(x^2\) &     &     \\ \hhline{~~~-----}
                &   &         &           & \(x^3\)   &           & + x & + 1 \\
                &   &         &         - & \(x^3\)   & + \(x^2\) & + x &     \\ \hhline{~~~~----}
                &   &         &           &           & \(x^2\)   &     & + 1 \\
                &   &         &           &         - & \(x^2\)   & + x & + 1 \\ \hhline{~~~~~---}
                &   &         &           &           &           &   x &    
\end{tabular}

= \(x^3 + x^2 + x + 1,\ R\ x\)

\section{Problem 3}

MI of 010 with IP \(x^3 + x + 1\) = \(x^2 + 1\)\\
\[\def\arraystretch{1.2}
\begin{array}{rlrrrrr}
&&&x^2&&+1&+R\ 1\\
\cline{2-6}
x&)&x^3&&+x&+1&\\
&&-x^3&&&&\\ \cline{3-6}
&&&&x&+1&\\
&&&&-x&&\\ \cline{5-6}
&&&&&1&
\end{array}
\]
\\
MI of 010 with IP \(x^3 + x^2 + 1\) = \(x^2 + x\)\\
\[\def\arraystretch{1.2}
\begin{array}{rlrrrrr}
&&&x^2&+x&&+R\ 1\\
\cline{2-6}
x&)&x^3&+x^2&&+1&\\
&&-x^3&&&&\\ \cline{3-6}
&&&x^2&&+1&\\
&&&-x^2&&&\\ \cline{4-6}
&&&&&1&
\end{array}
\]
\section{Problem 4}
Solved using program for Problem 6\\
With IP \(x^3 + x + 1\)

\begin{equation*}
  \begin{split}
    (x^2 + x + 1) + (x^2 + 1) &= x \\
    (x^2 + x + 1) - (x^2 + 1) &= x \\
    (x^2 + x + 1) * (x^2 + 1) &= x^2 + x \\
    (x^2 + x + 1) / (x^2 + 1) &= (x^2 + x + 1) * (x^2 + 1)^{-1} \mod (x^3 + x + 1) \\
    &= (x^2 + x + 1) * x \mod (x^3 + x + 1)\\
    &= (x^3 + x^2 + x) \mod (x^3 + x + 1)\\
    &= x^2 +1
  \end{split}
\end{equation*}



With IP \(x^3 + x^2 + 1\)

\begin{equation*}
  \begin{split}
    (x^2 + x + 1) + (x^2 + 1) &= x \\
    (x^2 + x + 1) - (x^2 + 1) &= x \\
    (x^2 + x + 1) * (x^2 + 1) &= 1 \\
    (x^2 + x + 1) / (x^2 + 1) &= \text{new calculations} \\
  \end{split}
\end{equation*}

\section{Problem 5}

Solved with our program for Problem 6.

\subsection{Binary Representations}

\(f(x) = 0xad = 1010\ 1101\) \\
\(g(x) = 0x0d = 0000\ 1101\) \\

\subsection{Multiplicative Inverses}

MI of 0xad = 0xe7 = \(x^7 + x^6 + x^5 + x^2 + x + 1\)\\
\\
MI of 0x0d = 0xe1 = \(x^7 + x^6 + x^5 + 1\)

\section{Problem 6}

See emailed code.

\end{document}
